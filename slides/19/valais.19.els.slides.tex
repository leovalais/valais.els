\documentclass[aspectratio=169]{beamer}
\usepackage[utf8]{inputenc}
\usepackage[T1]{fontenc}
\usepackage{graphicx}
\usepackage{tikz}
\usepackage{minted}
\usepackage{appendixnumberbeamer}
\usepackage{hyperref}
\usepackage{macosbox}
\usepackage{codebox}
\usepackage{apple_emoji}
\usepackage[normalem]{ulem}
\usepackage{forest}
% \usepackage[texcoord,grid,gridunit=mm,gridcolor=red!20,subgridcolor=gray!10]{eso-pic}

\usemintedstyle{colorful}
\usecodeboxmintedstyle{zenburn}
\newminted{cl}{autogobble,breaklines,escapeinside=||}
\setmintedcodebox{clcode}{title=Common Lisp,icon=\(\mathbf{\lambda}\),compact}
\newminted[smallclcode]{cl}{autogobble,breaklines,escapeinside=||,fontsize=\scriptsize}
\setmintedcodebox{smallclcode}{title=Common Lisp,icon=\(\mathbf{\lambda}\),compact}
\newminted[hlclcode]{cl}{autogobble,breaklines,escapeinside=||}
\setmintedcodebox{hlclcode}{headless,compact}

\usetheme{avalon}
\def\avalonprogressbar{1}
\def\avalondarkmintedstyle{monokai}%{zenburn}

\title{Implementing Baker's \texttt{SUBTYPEP} decision procedure\vspace*{-5mm}}
% \subtitle{European Lisp Symposium}
\date{April 1st, 2019}
\author{Léo Valais}
\institute{European Lisp Symposium}

\titlegraphic{\includegraphics[scale=.2]{lrde_epita.png}}

\renewcommand\code[1]{\texttt{#1}}
\newcommand\rarr{\ensuremath{\rightarrow}}
\newcommand\Rarr{\ensuremath{\Rightarrow}}
\newcommand\plholder[1]{\ensuremath{\langle {#1} \rangle}}
\newcommand\emoji[2][\tiny]{{#1#2}}
\newcommand\plus{{\color{watchOS-blue}\faPlus}}
\newcommand\minus{{\color{watchOS-red}\faMinus}}
\newcommand\good{{\color{watchOS-green}\faCheck}}
\newcommand\bad{{\color{watchOS-red}\faClose}}
\newcommand\tgood{{\color{watchOS-purple}\faThumbsOUp}}
\newcommand\tbad{{\color{watchOS-red}\faThumbsODown}}
\newcommand\maybe{{\color{watchOS-purple}\faQuestion}}

\begin{document}

\begin{frame}
\titlepage{}
\end{frame}

\newenvironment{sectionframe}[1]{%
  \begin{frame}[standout]
    \centering
    \Huge
    {\bf #1}

    {\usebeamercolor[bg]{secondary color}\noindent\sout{\hfill}}
    \bigskip

    \large
  }{\end{frame}}

\begin{sectionframe}{Motivations}
  Common Lisp type system, \code{subtypep} \\\& Baker's decision procedure
\end{sectionframe}

\begin{frame}[fragile]
  \frametitle{The Common Lisp type system}
  \begin{itemize}
  \item Types \rarr{} sets, subtypes \rarr{} subsets
  \item<3-> Types \rarr{} \emph{first class} values
  \end{itemize}

  \begin{itemize}
  \item<4-> \code{(subtypep \plholder A \plholder B)} $\equiv A \subseteq B?$
  \item<4-> Predicate function
  \end{itemize}

  \pause
  \medskip

  \begin{overprint}
    \onslide<2>
\begin{clcode}
(defun tr (M)
  (declare (type (array real (3 3)) M))
  (+ (aref M 0 0)
     (aref M 1 1)
     (aref M 2 2)))
\end{clcode}

    \onslide<5->
\begin{clcode}
(subtypep '(or my-class string (integer 0 (1024)))
          '(or super-class
               (array * 1)
               (unsigned-byte 10)))
\end{clcode}
  \end{overprint}

  \begin{popup}{.75}
    \onslide<6>
    \begin{macosbox}{}
      \begin{itemize}
      \item Type specifiers arbitrarily deep
      \item May take a while to retrun
      \end{itemize}

      \begin{alertblock}{Problem \#1 --- complex input}
        Arbitrarily complex input type specifiers
      \end{alertblock}
    \end{macosbox}
  \end{popup}
\end{frame}

\begin{frame}
  \frametitle{\code{subtypep} cannot always answer}
  \begin{itemize}
  \item \code{(satisfies \plholder{predicate})} $\equiv \left\{x \mid
      predicate(x)\right\}$
  \item \code{(satisfies oddp)} \rarr{} all odd numbers
    \pause
  \item \code{(subtypep '(satisfies oddp) '(satisfies evenp))}
    \pause
  \item \alert{halting problem} \rarr{} \code{subtypep} \emph{cannot} even answer
    \emoji{😱}
  \end{itemize}

  \pause
  \begin{alertblock}{Problem \#2 --- undecidability}
    \code{Subtypep} cannot answer for some type specifiers
  \end{alertblock}
\end{frame}


\begin{frame}
  \frametitle{\code{subtypep} return values}
  \[
    \code{(subtypep \plholder{A} \plholder{B})} =
    \begin{cases}
      \code{(T T)} &\rarr{} A \subseteq B \\
      \code{(NIL T)} &\rarr{} A \not\subseteq B \\
      \code{(NIL NIL)} &\rarr{} \text{\alt<1>{``undecidable''}{%
          \alert<2>{``I gave up, sorry \emoji{🤕}''}}}
    \end{cases}
  \]

  \begin{itemize}
  \item<1-> \code{(NIL NIL)} encodes \alt<1>{undecidability}
    {\sout{undecidability}\alert<2>{``input too complex''}}
  \item<3> Lack of reliability
  \item<3> Painful limit for some applications
    \begin{itemize}
    \item Newton's regular type expressions
    \item Newton's optimized \code{typecase} implementation
    \end{itemize}
  \end{itemize}
\end{frame}

\begin{frame}
  \frametitle{Baker's decision procedure}
  \setbeamercovered{highly dynamic}
  \begin{columns}
    \column{.45\paperwidth}
    \begin{itemize}
    \item[\plus] focus on result accuracy
    \item[\plus] \emph{never} returns \code{(NIL NIL)} when it is possible to
      answer
      \pause
    \item[\minus] paper difficult to read
    \item[\minus] not exhaustive
    \item[\minus] very few solutions about \code{satisfies}
    \end{itemize}

    \pause
    \column{.45\paperwidth}
    \begin{itemize}
    \item[\minus] no implementation available
    \item[\minus] exponential complexity (theoretical)
    \item[\maybe] efficiency% \rarr{} comparable to ``existing'' implementations
    \end{itemize}
  \end{columns}
\end{frame}

\begin{frame}
  \frametitle{Content}
  \begin{enumerate}
    \bf
    \setlength\itemsep{1em}
  \item Application using \code{subtypep}
  \item Baker's decision procedure
    \begin{enumerate}
      \setlength\itemsep{.5em}
      \it
    \item Pre-processing
    \item Types as bit-vectors
    \item Type specifier \rarr{} bit-vector expression
    \end{enumerate}
  \item Going further
  \end{enumerate}
\end{frame}

% \begin{sectionframe}{Application}
%   % Step by step application for some simple example
%   Use case of \code{subtypep}
% \end{sectionframe}

\begin{frame}[fragile]
  \frametitle{The problem}
  \begin{columns}
    \column{.55\paperwidth}
\begin{smallclcode}
(defclass point ()
  ((x :type number
      :initarg :x)
   (y :type number
      :initarg :y)
   (name :type string
         :initarg :name))
  (:metaclass json-serializable))

(json-serialize (make-instance 'point
                               :x -10
                               :y 3.2
                               :name "a1"))
\end{smallclcode}

    \pause

    \column{.4\paperwidth}
  \begin{mintedcodebox}[title=JSON serialization,icon=\faOpera,compact]
\begin{minted}[fontsize=\scriptsize]{json}
{
  "X": -10,
  "Y": 3.2,
  "NAME": "a1"
}
\end{minted}
  \end{mintedcodebox}

  \pause

\begin{smallclcode}
(deftype json ()
  '(or number
       string
       (and symbol
            (not keyword))
       list
       hash-table))
\end{smallclcode}
\end{columns}
\end{frame}

\begin{frame}[fragile]
  \frametitle{Our \code{employee} class}
  \begin{itemize}
  \item 2 slots \Rarr{} 2 calls to \code{subtypep}
  \item Trigger error if one fails
  \end{itemize}
\begin{clcode}
(defclass employee ()
  ((name :type (or string
                   (and symbol
                        (not keyword))
                   unsigned-byte))
   (part-time-p boolean))
  (:metaclass json-serializable))
\end{clcode}
\end{frame}

\begin{sectionframe}{Baker's decision procedure}
  Application of our implementation to check\\\smallskip
  $\code{employee.name} \subseteq \code{json}$
\end{sectionframe}

\begin{frame}[fragile]
  \frametitle{Pre-processing steps}
  \begin{columns}
    \column{.65\textwidth}
    \begin{overprint}
      \onslide<1>
\begin{clcode}
(subtypep '(or string
               (and symbol
                    (not keyword))
               unsigned-byte)
          'json)
\end{clcode}

      \onslide<2>
\begin{clcode}
(subtypep '(or string
               (and symbol
                    (not keyword))
               unsigned-byte)
          '(or number
               string
               (and symbol
                    (not keyword))
               list
               hash-table))
\end{clcode}

      \onslide<3->
\begin{clcode}
(subtypep
 '(AND (or string
           (and symbol
                (not keyword))
           unsigned-byte)
        (NOT (or number
                 string
                 (and symbol
                      (not keyword))
                 list
                 hash-table)))
 NIL)
\end{clcode}
    \end{overprint}

    \column{.35\textwidth}
    \begin{itemize}
    \item<2-> Alias expansion
    \item<3-> $P \subseteq Q \Rarr P \cap \neg Q = \emptyset$
    \end{itemize}
  \end{columns}
\end{frame}

\newcommand\bs{\ensuremath{\mathcal B}}
\begin{frame}[fragile]
  \frametitle{Bit-vector type representation}
  \pause
  \begin{columns}
    \column{.6\textwidth}
  Types represented as bit-vectors $\bs_P$

  \[
    \renewcommand\bs[1]{\ensuremath{\mathcal B_{\scriptsize\code{#1}}}}
    \bordermatrix{
      &\code{t}&\code{nil}&\code{sym}&\code{"str"}&\cdots&\code{\scriptsize(l i s t)}\cr
      \bs{nil}     &0     &0     &0     &0     &\cdots&0\cr
      \bs{t} &1     &1     &1     &1     &\cdots&1\cr
      \bs{null}     &0     &1     &0     &0     &\cdots&0\cr
      \bs{symbol}     &1     &1     &1     &0     &\cdots&0\cr
      \bs{string}     &0     &0     &0     &1     &\cdots&0\cr
      \vdots  &\vdots&\vdots&\vdots&\vdots&\ddots&\vdots\cr
      \bs{list}     &0     &1     &0     &0     &\cdots&1\cr
    }
  \]

    \column<3>{.4\textwidth}
    \begin{block}{Properties (bitwise)}
      \vspace*{-1em}
      \begin{align*}
        \bs_{P \cup Q} &= \bs_P \vee \bs_Q\\
        \bs_{P \cap Q} &= \bs_P \wedge \bs_Q\\
        \bs_{\overline P} &= \neg\bs_P
      \end{align*}
    \end{block}
  \end{columns}

  \begin{popup}{.75}
    \onslide<2>
  \end{popup}
\end{frame}

\begin{frame}[fragile]
  \frametitle{Back to our problem}
\begin{clcode}
(subtypep '(and (or string
                    (and symbol
                         (not keyword))
                    unsigned-byte)
                (not (or number
                         string
                         (and symbol
                              (not keyword))
                         list
                         hash-table)))
          nil)
\end{clcode}

  \begin{popup}{.85}
    \onslide<2-3>
    \begin{macosbox}{Bit-vector expression reduction}
      \centering
      \newcommand\inter{\alt<2>{$\cap$}{$\wedge$}}
      \newcommand\union{\alt<2>{$\cup$}{$\vee$}}
      \newcommand\compl{\alt<2>{$\phantom{ }^C$}{$\neg$}}
      \newcommand\co[1]{\alt<2>{\code{#1}}{\bs{\code{\scriptsize#1}}}}
      \begin{forest}
        [\inter
         [\union
          [\co{string}]
          [\inter
           [\co{symbol}]
           [\compl [\co{keyword}]]]
          [\co{unsigned-byte}]]
         [\compl
          [\union
           [\co{number}]
           [\co{string}]
           [\inter [\co{symbol}] [\compl [\co{keyword}]]]
           [\co{list}]
           [\co{hash-table}]]]]
      \end{forest}
    \end{macosbox}
  \end{popup}
\end{frame}

\begin{frame}[fragile]
  \frametitle{\code{employee} verification}
  \begin{columns}
    \column{.65\textwidth}
\begin{hlclcode}
(defclass employee ()
  ((name :type (or string
                   (and symbol
                        (not keyword))
                   unsigned-byte))
   (part-time-p boolean))
  (:metaclass json-serializable))
\end{hlclcode}

\begin{hlclcode}
(subtypep '(or string
               (and symbol
                    (not keyword))
               unsigned-byte)
          'json)
\end{hlclcode}

    \column{.35\textwidth}
    \begin{itemize}
    \item[\good] \code{employee.name}
    \item<2->[\alt<2>{\maybe}{\good}] \code{employee.part-time-p}
      % \begin{itemize}
      % \item<3-> $\code{boolean} \subseteq \code{symbol} \Rarr \code{boolean}
      %   \subseteq \code{json}$
      % \end{itemize}
    \end{itemize}

    \pause
    \bigskip
    \uncover<4>{%
      \begin{exampleblock}{Conclusion}
        \code{employee} is JSON-compatible! \emoji{🎉}
      \end{exampleblock}}
  \end{columns}
\end{frame}

\begin{sectionframe}{CLOS classes \& \code{member} type specifiers}
  Choosing representative elements right
\end{sectionframe}

\begin{frame}
  \frametitle{CLOS classes}
  \setbeamercovered{highly dynamic}

  \begin{itemize}
  \item Issue \rarr{} find a representative instance
  \item Cannot use \code{make-instance} \rarr{} possible side-effects
    \pause
  \item Baker's solution
    \begin{itemize}
    \item \emph{hook into \code{defclass} implementation}
    \item[\tiny\minus] not portable
    \item[\tiny\minus] maybe not trivial
    \end{itemize}
  \end{itemize}

  \pause

  \begin{itemize}
  \item Our solution \rarr{} the Meta Object Protocol
    \begin{itemize}
    \item \emph{register class prototypes \rarr{} ``fake'' instances}
    \item[\tiny\plus] portable (for implementations supporting the MOP)
    \item[\tiny\plus] easier to implement
    \item[\tiny\plus] packageable
    \end{itemize}
  \end{itemize}
\end{frame}

\begin{frame}[fragile]
  \frametitle{\code{member} type specifiers}
  \begin{itemize}
  \item Explicitly provide type's elements
  \item \code{(member \plholder A \plholder B \plholder C)} $\equiv \{A, B, C\}$
  \item ``Anonymous'' types
  \item Bit-vector $\bs_{\scriptsize\code{(member \plholder A \plholder B \plholder
        C)}}$
    \begin{enumerate}
    \item add $A, B, C$ as representatives
    \item $\bs_{\scriptsize\code{(member \plholder A \plholder B \plholder
        C)}} = \bs_{\{A\}} \vee \bs_{\{B\}} \vee \bs_{\{C\}}$
    \end{enumerate}
  \end{itemize}
\end{frame}

\begin{frame}[standout]
  \frametitle{Conclusion}
  \begin{itemize}
  \item \code{subtypep} unreliability
  \item Baker's decision procedure
    \begin{itemize}
    \item no implementation given
    \item many details missing
    \item seems elegant and powerful
    \end{itemize}
  \item Our implementation
    \begin{itemize}
    \item incomplete \& experimental
    \item motivating accuracy \& performance measures
    \end{itemize}
  \item Future work
    \begin{itemize}
    \item implement missing type specifiers (\code{array} \& \code{complex})
    \item find solutions for \code{cons} \& \code{satisfies}
    \item open source the implementation!
    \end{itemize}
  \end{itemize}
\end{frame}

\begin{sectionframe}{\it Thanks for listening! \emoji[\normalsize]{😃}}
  \it Any question?
\end{sectionframe}

% \begin{frame}[allowframebreaks]
%   \bibliographystyle{plain}
%   \nocite{bib:ansi.94.cl}
%   \nocite{baker1992}
%   \nocite{newton.18.els}
%   \nocite{newton.18.phd}
%   \nocite{gcl-devel.cons}
%   \nocite{valais.19.els}
%   \nocite{verna.06.ecoop}
%   \bibliography{../common}
% \end{frame}

\end{document}
